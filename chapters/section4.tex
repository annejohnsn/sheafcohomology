\newpage
\section*{Section 4}
4.1 (a) Show that with the notation as above, $\Gamma(-, \pi)$ is indeed isomorphic to the sheaffication of $P$. \\
4.1(b) Show that the $coker(\alpha)$ has the following universal property: given any other sheaf $H$, a map $\chi: G \rightarrow H$ factors through $coker(\alpha)$ if and only if $\chi \circ \alpha = 0$.\\
4.1(c) Show that $coker(\alpha)_x = coker(\alpha_x)$.

\newpage
4.2 Show that \textbf{Ab(X)} is an Abelian category with kernel, cokernel and sum defined as above.

\newpage
4.3 Let $0 \rightarrow A \hookrightarrow I^0 \rightarrow I^1 \rightarrow ...$ and $0 \rightarrow B \rightarrow B \hookrightarrow J^0 \rightarrow J^1 \rightarrow J^2 \rightarrow ...$ be injective resolutions of $A$ and $B$ respectively. Show that a map $\phi: A \rightarrow B$ extends to a map of complexes:
\[
\begin{tikzcd}
0 \arrow[r] \arrow[d, equal]
& A \arrow[r] \arrow[d, "\phi", dashed]
& I^0 \arrow[r] \arrow[d, "\phi^0", dashed]
& I^1 \arrow[r] \arrow[d, "\phi^1", dashed]
& I^2 \arrow[r] \arrow[d, "\phi^2", dashed]
& ...
\\
0 \arrow[r]
& B \arrow[r]
& J^0 \arrow[r]
& J^1 \arrow[r]
& J^2 \arrow[r]
& ...
\end{tikzcd}
\]

%\[
%\begin{tikzcd}
%	 E \arrow[d, "p"]
%	& A \arrow[r] \arrow[d, "\phi"] 
%	& I^0 \arrow[r] \arrow[d, "\phi"] 
%	& I^1 \arrow[r] \arrow[d, "\phi"] 
%	& I^2 \arrow[r] \arrow[d, "\phi"]
%	& "..." 
%	\\
%	0 \arrow[r] \arrow[d, equal]
%	& B \arrow[r] \arrow[d, "\phi"] 
%	& J^0 \arrow[r] \arrow[d, "\phi"] 
%	& J^1 \arrow[r] \arrow[d, "\phi"] 
%	& J^2 \arrow[r] \arrow[d, "\phi"]
%	& "..." 
%\end{tikzcd}
%\]

4.3 (b) Show that a map of complexes as above induces a homomorphism of cohomology groups 
\[ H^n(X;A) \rightarrow H^n(X,B)\] 

\newpage
4.4(a) Let $f: Y \rightarrow X$ be a map of topological spaces. Show that there is a natural isomorphism $\Gamma_Y \rightarrow \Gamma_X \circ f_*$, 
\[
\begin{tikzcd}
Ab(Y) \arrow[rr,"f_*"] \arrow[rd, "\Gamma_Y"'] & & Ab(X) \arrow[ld, "\Gamma_X"]\\
& \textbf{Abelian Groups} 
\end{tikzcd}
\]
4.4 (b) Show that a natural isomorphism between left exact functors $\tau: T_1 \rightarrow  T_2$ induces a natural isomorphism $R^nT_1 \rightarrow R^nT_2$ for each $n$.

