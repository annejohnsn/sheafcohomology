\documentclass{article}

\usepackage[all]{xy}
\usepackage{amsfonts, amssymb}
\usepackage{graphics} 
\newcommand{\bfA}{\mathbf{A}}
\newcommand{\bfB}{\mathbf{B}}
\newcommand{\bfC}{\mathbf{C}}
\newcommand{\bfD}{\mathbf{D}}
\newcommand{\Hom}{\mbox{\rm Hom}}
\newcommand{\Et}{\mbox{\rm Et}}
\oddsidemargin 0.75cm
\textwidth 15cm
\newcommand{\zed}{\mathbb Z}
\begin{document}
\begin{center}
{\bf Sheaf Cohomology Final Exam for Anne Johnson, September 2018}
\end{center}

\begin{enumerate}
\item
Let $\varphi\colon F\to G$ be a morphism of sheaves on a space $X$.
\begin{enumerate}
 \item Show that the induced map $\Et(\varphi)\colon\Et(F)\to\Et(G)$ between the associated 
\'etale spaces over $X$ is surjective if and only if for every open set $U$ of $X$, and every section $s\in G(U)$,
there exists an open cover $(U_i)_{i\in I}$ of $U$ and sections $t_i\in F(U_i)$ such that $\varphi(t_i)=s|_{U_i}$ in $G(U_i)$ for all $i\in I$.
\item Give an example of a surjective morphism of sheaves $\varphi\colon F\to G$, and an open set $U$ such that $\varphi_U\colon F(U)\to G(U)$ is not surjective.
\end{enumerate}
 
 \item 
Let $[0, 1]$ denote the closed interval and $(0, 1)$ denote the open
interval and $i: (0, 1) \to [0, 1]$ the inclusion map. Write $\Delta_{(0,1)}\zed$
and $\Delta_{[0,1]}\zed$ for the constant sheaves on $(0,1)$ and $[0,1]$ respectively. Let
$$\varphi: i_!\Delta_{(0,1)}\zed \to \Delta_{[0,1]}\zed$$
be the adjunct of the identity map
$$\Delta_{(0,1)}\zed \to \Delta_{(0,1)}\zed = i^*\Delta_{[0,1]}\zed.$$
Calculate the cokernel of $\varphi$. Give it both as an \'etale space and
as a sheaf defined as a functor.
\item
Describe the direct sum $F\oplus G$ of two sheaves $F$ and $G$ on a space $X$.
Then show that $H^n(X;F\oplus G)\cong H^n(X;F)\oplus H^n(X;G)$; i.e., sheaf cohomology commutes with taking direct sums.

\item 
Given an abelian group $G$ and a topological space $X$ with a point $p\in X$, define a presheaf $G_p$ on $X$ by
$$
G_p(U)=\left\{\begin{array}{ll}G,&\mbox{ if }p\in U,\\0,&\mbox{ if }p\not\in U\end{array}\right.
\quad \mbox{and}\quad \rho_{VU}(a)=\left\{\begin{array}{ll}a&\mbox{ if }p\in V\subseteq U,\\
                                                                                                                                           0,&\mbox{ if }p\not\in V
                                                                                                                                          \end{array}\right.
$$
for $V\subseteq U\subseteq X$ open sets. 
\begin{enumerate}
\item Verify that this defines a sheaf on $X$. 
\item Describe the stalks at each point of the space.
\item Let $F$ be a sheaf on a one-point space $X=\{P\}$. Show that $H^1(X;F)=0$.
\item Let $X$ be an arbitrary space with $p\in X$ and $G_p$ as defined above.
Show that $H^1(X;G_p)=0$.
\end{enumerate}

\item
The torus $T$ is obtained as the quotient space of $[0,1]\times[0,1]$ by the equivalence relation generated by: 
$(x,0)\sim (x,1)$ for all $x\in [0,1]$ and $(0,y)\sim(1,y)$ for all $y\in [0,1]$.

The Klein bottle $K$ is obtained as the quotient space of $[0,1]\times[0,1]$ by the equivalence relation generated by: 
$(x,0)\sim (x,1)$ for all $x\in [0,1]$ and $(0,y)\sim(1,1-y)$ for all $y\in [0,1]$.
\begin{enumerate}
\item
Calculate the cohomology groups of the torus $T$ with values in the constant sheaf $\Delta_T\zed$. (Hint: use Corollary 7.3.2.)
\item
Use a Mayer-Vietoris sequence to calculate the cohomology groups of the Klein bottle with values in the constant sheaf $\Delta_K\zed$. 
\item
Let $S$ be the middle circle of the torus, given by the image of the segment $\{(x,0)|\,x\in[0,1]\}$ in the quotient space. Give the long exact sequences for the pair $(T,S)$ 
and for the relative cohomology with coefficients in $\Delta_T\zed$ and calculate as many of the groups as you can.
\item
Let $S'\subset K$ be the image of $\{(x,0)|\,x\in[0,1]\}$ in the quotient space $K$.
Use the long exact sequence from Proposition 9.4.1 and fill in as many groups as you can to find the cohomology with compact support for 
the open annulus and the open M\"obius strip.
\end{enumerate}


\end{enumerate}

\end{document}