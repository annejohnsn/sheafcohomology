\newpage
\section*{Section 2}

\begin{definition*}
		($germ_x(s)$)\\ If $P$ is a presheaf on a space $X$, then for all $x \in X$ and all $U \ni x$, we define an equivalence relation on the pairs $(U,s)$ where $s$ is an element of $P(U)$ by $(U,s) \sim (V,t)$ if and only if $\exists W \subseteq U \cap V$, an open neighborhood of $x$, such that $s \mid_W = t \mid_W$. The equivalence class of $(U,s)$ is called the germ of s at x and denoted $germ_x(s)$.
\end{definition*}

\begin{definition*}
	(stalk of $P$)\\ The stalk of $P$ at $x \in X$, a presheaf on a space $X$, is the set of germs at $x$, $\{germ_x(s) \mid U \ni x$ and $s \in P(U)\}$, denoted by $P_x$.
\end{definition*}

\begin{definition*}
	(\'etale space) \\ If $P$ is a presheaf on $X$, the \'etale space of $P$ is the disjoint union over $x \in X$ of the stalks of $P$, denoted by $Et(P)$. It is a topological space with basis open sets given by $B(s) = \{germ_x(s) \mid x \in U\}$ for some open $U$ containing $x$. 
\end{definition*}

\begin{definition*}
	(\'etale map)\\ A map $\pi: E \rightarrow X$ between topological spaces is called an \'etale map or a local homeomorphism if for each $e \in E$ there exists an open neighborhood $B$ of e and $U$ of $\pi(B)$ such that $\pi_B$ is a homeomorphism. 
\end{definition*}
	
\begin{definition*}
	(associated sheaf)\\ Suppose $P$ is a presheaf on $X$. Define a map $\pi_P: Et(P) \rightarrow X$ by $germ_x(s) \mapsto x$. The sheaf of sections of this map, $\Gamma Et(P)$, is the associated sheaf (sheaffication) of P.
\end{definition*}

\begin{definition*}
	($\eta_P$)\\ Let $P$ be a presheaf on X. We define $\eta_P: P \rightarrow \Gamma Et(P)$ by $\eta_{p,U}(s) = x \mapsto germ_x(s)$. That is, $\eta$ takes a section of $P(U)$ and sends it to the function $\sigma: U \rightarrow Et(P)$ defined by $x \in U \mapsto germ_x(s)$ (note that for any section in $P(U)$ we can ask what the equivalence class of $s$ is at any $x \in U$ - look at the restriction of $s$ to open nbhds of $x$). 
\end{definition*}

\begin{definition*}
	($\epsilon_E$)\\ Let $f: E \rightarrow X$ be any map. We define $\epsilon_E$ from the \'etale space of $\Gamma E$ to $E$ by $\epsilon_E(germ_x(s)) = s(x)$ ($s$ is a map from $U \rightarrow E$ such that $f \circ s = id_U$, $germ_x(s)$ describes the restriction of $s$ to an open nbhd of $x$ for every $x$ in $U$
\end{definition*}