 \newpage
\section*{Section 4}
4.1 (a) Show that with the notation as above, $\Gamma(-, \pi)$ is indeed isomorphic to the sheaffication of $P$. \\

We first fix some notation. Suppose $U \subseteq X$ and let $s \in G(U)$ be a section. Let $\overline{s}$ be the equivalence class of $s$ in $G(U)/im(\alpha_U) = coker(F(U) \xrightarrow{\alpha_U} G(U))$, so that  points in the \'etale space of $P$ are  denoted $germ_x(\overline{s})$. We denote by $\overline{germ_x(s)}$ the equivalence class of $germ_x(s)$ under the relation $R$ given above, where $germ_x(s) \in Et(G)$. \\

To show $\Gamma(-, \pi) \cong P$, we define a map $\phi: \Gamma(-, \pi) \rightarrow P$ and show it induces an isomorphism on the stalks. For $U \subseteq X$ and $\sigma \in \Gamma(-,\pi)(U)$, define $\phi_U:\Gamma(-,\pi)(U) \rightarrow P(U)$ by $\phi_U(\sigma)=\tau$, where $\sigma$ is a function $x \mapsto \overline{germ_x(s)}$ and $\tau$ is a function $x \mapsto germ_x(\overline{s})$ for some $s \in G(U)$. Naturality of $\phi$ is straightforward, since restriction maps are just restrictions of functions. \\

To see that $\phi$ is well defined, suppose $germ_x(t)$ is also a representative of $\overline{germ_x(s)}$. Then $\exists W \ni x$ and $f \in F(W)$ such that $s\mid_W - t\mid_W = \alpha_W(f)$. This however gives $germ_x(\overline{t}) = germ_x(\overline{s})$, since $W$ is then a neighborhood for which $\overline{s\mid_W} = \overline{t\mid_W}$, so that they are equal as germs. \\

It remains to show that $\phi$ induces an isomorphism on the stalks. We show $\phi_x: \Gamma(-, \pi)_x \rightarrow P_x$ is injective and surjective. Fix an $x \in X$ and suppose $\phi(\overline{germ_x(a)})=germ_x(\overline{a})=\phi(\overline{germ_x(b)})$. We have to show $\exists W \ni x$ and $f \in F(W)$ such that $a\mid_W - b\mid_W = \alpha_W(f)$.
To see that $\phi_x$ is surjective.


\newpage


Let $\overline{germ_x(s)}$ be an element of $Et(G)/R$, and let $germ_x(s)$ be a representative, so that if $germ_x(t)$ is another representative of $\overline{germ_x(s)}$, then $\exists W \ni x$ and $f \in F(W)$ such that $s\mid_W - t\mid_W = \alpha_W(f)$, where $s$ and $t$ are representatives of $germ_x(s)$ and $germ_x(t)$ respectively.\\

Let $germ_x(a)$ be an element of $P_x$, represented by $(a,U)$ for some $U \ni x$ and $a \in P(U)  = coker(F(U) \xrightarrow{\alpha_U} G(U)) = G(U)/im(\alpha_U)$. If $(b,V)$ is another representative of $germ_x(a)$, then there exists $W \ni x$ such that $a\mid_W = b\mid_W$, i.e., $a\mid_W$ and $b\mid_W$ are in the same equivalence class of $G(W)/im(\alpha_W)$, i.e., $\exists f \in F(W)$ such that $s\mid_W - t\mid_W = \alpha_W(f)$. Then elements of $G_x/R$ and $P_x$ are both correspond to elements of $G_x$ identified  via the same relation.\\


Let $\overline{germ_x(s)}$ be an element of $G_x/R$, and let $germ_x(s)$ be a representative, so that if $germ_x(t)$ is another representative of $\overline{germ_x(s)}$, then $\exists W \ni x$ and $f \in F(W)$ such that $s\mid_W - t\mid_W = \alpha_W(f)$, where $s$ and $t$ are representatives of $germ_x(s)$ and $germ_x(t)$ respectively.\\

Let $germ_x(a)$ be an element of $P_x$, represented by $(a,U)$ for some $U \ni x$ and $a \in P(U)  = coker(F(U) \xrightarrow{\alpha_U} G(U)) = G(U)/im(\alpha_U)$. If $(b,V)$ is another representative of $germ_x(a)$, then there exists $W \ni x$ such that $a\mid_W = b\mid_W$, i.e., $a\mid_W$ and $b\mid_W$ are in the same equivalence class of $G(W)/im(\alpha_W)$, i.e., $\exists f \in F(W)$ such that $s\mid_W - t\mid_W = \alpha_W(f)$. Then elements of $G_x/R$ and $P_x$ are both correspond to elements of $G_x$ identified  via the same relation.\\



\newpage
4.1(b) Show that the $coker(\alpha)$ has the following universal property: given any other sheaf $H$, a map $\chi: G \rightarrow H$ factors through $coker(\alpha)$ if and only if $\chi \circ \alpha = 0$.\\

(Forward direction) Let $\alpha:F \rightarrow G$ and $\chi:G \rightarrow H$ and suppose $\chi \circ \alpha =0$. Then for $U \in X$, $ker(\chi_U) = im(\alpha_U)$. By the first isomorphism theorem, $\chi_U(G) \cong G(U)/ker(\chi_U) = G(U)/im(\alpha_U) = coker(F(U) \xrightarrow{\alpha_U} G(U)) $, so that $\chi = g \circ f$, where $f:G \rightarrow coker(\alpha)$ is the projection map and $g:coker(\alpha) \rightarrow H$ is an isomorphism.\\

(Backwards direction) Let $\alpha:F \rightarrow G$ and $\chi:G \rightarrow H$ and suppose $\chi= g \circ f$ for some $f:G \rightarrow coker(\alpha)$ and $g:coker(\alpha) \rightarrow H$. Note that $f \circ \alpha = 0$, since $f_U$ has codomain $coker(F(U) \xrightarrow{\alpha_U} G(U)$ for $U \in X$. Then $\chi \circ \alpha = g \circ f \circ \alpha = g \circ 0 =0$.

\newpage
4.1(c) Show that $coker(\alpha)_x = coker(\alpha_x)$.\\

Let $\alpha: F \rightarrow G$ for $F$ and $G$ Abelian sheaves on $X$ and let $P$ be the presheaf given by $P(U)=coker(F(U) \xrightarrow{\alpha_U} G(U))$. Since $\pi_P: Et(P) \rightarrow X$ is \'etale, $Et\Gamma EtP \cong EtP$, so the left-hand side is isomorphic to $P_x$. On the right-hand side, note that $coker(\alpha_x)=coker(F_x \xrightarrow{\alpha_x} G_x) = G_x/Im(\alpha_x)$, so that two points are equivalent in $coker(\alpha_x)$ iff they are both points in $G_x$, say $germ_x(s)$ and $germ_y(t)$, with $germ_x(s) - germ_y(t) \in Im(\alpha_X)$, ie $\exists W \ni x$ and $f \in F(W)$ such that $s\mid_W - t\mid_W = \alpha_W(f)$. This is exactly the relation given by $R$ in the notes, so that the right-hand side is isomorphic to $G_x/R$. Then exercise 4.1(a) implies $coker(\alpha)_x \cong coker(\alpha_x)$.

\newpage
4.2 Show that \textbf{Ab(X)} is an Abelian category with kernel, cokernel and sum defined as above.\\

We first show that \textbf{Ab(X)} has a $0$ object. Let $\Delta \{*\}$ be the constant sheaf on a singleton set. Then for $U \subseteq X$,  $\Delta \{*\}(U)$ is the trivial group and every restriction map is the identity, so $\Delta \{*\} \in \textbf{Ab(X)}$. Moreover, for $A \in \textbf{Ab(X)}$, there is exactly one map from $A(U) \rightarrow \Delta \{*\}(U)$ for every open $U \subseteq X$, namely the constant $0$-map, and there is also exactly one map from $\Delta \{*\}(U) \rightarrow A(U)$ since group homomorphism must send identity to identity. Then $\Delta \{*\}$ is both initial and terminal, and so serves as the $0$ element in \textbf{Ab(X)}. We can now define the $0$ morphism in \textbf{Ab(X)} as the map defined for $A,B \in \textbf{Ab(X)}$ and $U \in X$, open, by $a \in A(U) \mapsto 0_{B(U)}$.\\

Let us define the sum of two Abelian sheaves $A$ and $B$ by $A(U) \oplus B(U)$ for $U$ open in $X$, as in the notes. Then this gives the biproduct of $A$ and $B$ in \textbf{Ab(X)}. We know from the notes that $A \oplus B$ is a sheaf and we observe that for $U \subseteq X$, $A(U) \oplus B(U)$ has an Abelian group structure: let $(a,b)$ and $(a',b')$ be elements of $A(U) \oplus B(U)$ for some $U$. Then $(a,b) + (a',b') = (a +a', b+b') = (a'+a,b'+b) = (a',b') + (a,b) \in A(U) \oplus B(U)$, since $A(U)$ and $B(U)$ are Abelian groups. This also implies $(a,b) +(0_{A(U)}, 0_{B(U)}) = (a,b) = (0_{A(U)}, 0_{B(U)}) + (a,b)$. To see that this is a biproduct, let $p_j: A_1 \oplus A_2 \rightarrow A_j$ be the map defined on $U \subseteq X$ as the projection map $p_{U,j}: A_1(U) \oplus A_2(U) \rightarrow A_j(U)$ and $i_j$ be the map defined on $U \subseteq X$ as the injection map $i_j(U): A_j(U) \rightarrow A_1(U) \oplus A_2(U) $ for $j \in \{1,2\}$ and $A_j \in \textbf{Ab(X)}$. Then since $p_{U,j}$ and $i_{U,j}$ give the product and coproduct on each $A_1(U) \oplus A_2(U)$, viewed as elements in the category of Abelian groups, their collection gives the product and coproduct on $A_1 \oplus A_2$ (given a map into or out of the biproduct, find a new map that factors through $p_j$ or $i_j$ respectively, by taking the collection of maps implied by the universal property of each $p_{U,j}$ or $i_{U,j}$).\\

To see that \textbf{Ab(X)} has kernels, note that if $A,B \in \textbf{Ab(X)}$ and $\phi: A \rightarrow B$, then $\phi$ has a kernel when everything is viewed in the category $\textbf{Sh(X)}$. It remains only to check that $ker(\phi)(U) = \{f \in A(U) \mid \phi_U(f) = 0_{B(U)}\}$ is an Abelian group for all $U$, open in $X$. But $ker(\phi)(U) = ker(\phi_U)$ is the kernel of the group homomorphism $\phi_U$, and so must itself be a subgroup of $A(U)$ for all $U \in X$. Then $ker(\phi) \in \textbf{Ab(X)}$. Likewise, the cokernel of $\phi$ exists in \textbf{Sh(X)}, and so we need only show $coker(\phi)(U)$ is an Abelian group, $\forall U$.

\newpage
4.3 Let $0 \rightarrow A \hookrightarrow I^0 \rightarrow I^1 \rightarrow ...$ and $0 \rightarrow B \rightarrow B \hookrightarrow J^0 \rightarrow J^1 \rightarrow J^2 \rightarrow ...$ be injective resolutions of $A$ and $B$ respectively. Show that a map $\phi: A \rightarrow B$ extends to a map of complexes:
\[
\begin{tikzcd}
0 \arrow[r] \arrow[d, equal]
& A \arrow[r] \arrow[d, "\phi"]
& I^0 \arrow[r] \arrow[d, "\phi^0", dashed]
& I^1 \arrow[r] \arrow[d, "\phi^1", dashed]
& I^2 \arrow[r] \arrow[d, "\phi^2", dashed]
& ...
\\
0 \arrow[r]
& B \arrow[r]
& J^0 \arrow[r]
& J^1 \arrow[r]
& J^2 \arrow[r]
& ...
\end{tikzcd}
\]

Since $0 \rightarrow A \hookrightarrow I^0 \rightarrow I^1 \rightarrow ...$ is a resolution, in particular exact, for each $n$, we have an injective map from $A \rightarrow I^n$ given by $d^n$. Since $\phi: A \rightarrow B$, we also have for each $n$, a map from $A \rightarrow J^n$ given by $d^n \circ \phi$. Then injectivity of $J^n$ implies the required map $\phi^n:I^n \rightarrow J^n$: 

\[
\begin{tikzcd}
0 \arrow[r] 
& A \arrow[r, "d^n"] \arrow[d, "d^n \circ \phi"']
& I^n \arrow[ld, "\phi^n", dashed]
\\
& 
J^n 
\end{tikzcd}
\]

\newpage
4.3 (b) Show that a map of complexes as above induces a homomorphism of cohomology groups 
\[ H^n(X;A) \rightarrow H^n(X,B)\] 

Note that the above map of chain complexes implies a map for $U \in X$:

\[
\begin{tikzcd}
0 \arrow[r] \arrow[d, equal]
& A(U) \arrow[r] \arrow[d, "\phi_U"]
& I^0(U) \arrow[r] \arrow[d, "\phi^0_U", dashed]
& I^1(U) \arrow[r] \arrow[d, "\phi^1_U", dashed]
& I^2(U) \arrow[r] \arrow[d, "\phi^2_U", dashed]
& ...
\\
0 \arrow[r]
& B(U) \arrow[r]
& J^0(U) \arrow[r]
& J^1(U) \arrow[r]
& J^2(U) \arrow[r]
& ...
\end{tikzcd}
\]

which for $U = X$ gives the map:

\[
\begin{tikzcd}
0 \arrow[r] \arrow[d, equal]
& \Gamma A \arrow[r] \arrow[d, "\Gamma \phi"]
& \Gamma I^0 \arrow[r] \arrow[d, "\Gamma \phi^0", dashed]
& \Gamma I^1 \arrow[r] \arrow[d, "\Gamma \phi^1", dashed]
& \Gamma I^2 \arrow[r] \arrow[d, "\Gamma \phi^2", dashed]
& ...
\\
0 \arrow[r]
& \Gamma B \arrow[r]
& \Gamma J^0 \arrow[r]
& \Gamma J^1 \arrow[r]
& \Gamma J^2 \arrow[r]
& ...
\end{tikzcd}
\]

Since $H^n(X;A) = ker(\Gamma I^n \rightarrow \Gamma I^{n+1})) /im (\Gamma I^{n-1} \rightarrow \Gamma I^n) \subseteq \Gamma I^n$ and likewise $H^n(X;B) = ker(\Gamma J^n \rightarrow \Gamma J^{n+1})) /im (\Gamma J^{n-1} \rightarrow \Gamma J^n) \subseteq \Gamma J^n$, the maps $\Gamma \phi^n: \Gamma I^n \rightarrow \Gamma J^n$ implies a homomorphism $\Gamma \phi^n\mid_{H^n(X,A)}: H^n(X;A) \rightarrow \Gamma J^n$. It remains only to check that $im(\Gamma \phi^n\mid_{H^n(X,A)}) \subseteq H^n(X,B)$. Let $\alpha \in H^n(X;A)$. Then $\alpha \in ker(d^{n}:\Gamma I^n \rightarrow \Gamma I^{n+1}) $, so $\Gamma\phi^{n+1} \circ d^{n} (\alpha) =0$. Since the above diagram commutes, this implies $d^n \circ \Gamma \phi^n (\alpha) =0$, so that $\Gamma \phi^n (\alpha) \in ker(d^n:\Gamma J^n \rightarrow \Gamma J^{n+1})$, which implies $\Gamma \phi^n (\alpha) \in H^n(X;B)$. 

\newpage
4.4(a) Let $f: Y \rightarrow X$ be a map of topological spaces. Show that there is a natural isomorphism $\Gamma_Y \rightarrow \Gamma_X \circ f_*$, 
\[
\begin{tikzcd}
Ab(Y) \arrow[rr,"f_*"] \arrow[rd, "\Gamma_Y"'] & & Ab(X) \arrow[ld, "\Gamma_X"]\\
& \textbf{Abelian Groups} 
\end{tikzcd}
\]

We must find $\eta_A$ and $\eta_B$,  isomorphisms, such that the following commutes for all $A,B \in Ab(Y)$ and $\phi:A \rightarrow B$: 

\[
\begin{tikzcd}
\Gamma_Y(A) \arrow[r, "\Gamma_Y \phi"] \arrow[d, "\eta_A"']
&  \Gamma_Y(B) \arrow[d, "\eta_B"] \\
\Gamma_X  \circ f_*(A) \arrow[r, "\Gamma_x \circ f_* \phi"']
& \Gamma_X  \circ f_*(B)
\end{tikzcd}
\]

By the definition of $\Gamma_{\_}$ and $f_*$, this is the same as finding isomorphisms $\eta_A$ and $\eta_B$ such that the following commutes:

\[
\begin{tikzcd}
A(Y) \arrow[r, "\phi_Y"] \arrow[d, "\eta_A"']
&  B(Y) \arrow[d, "\eta_B"] \\
A(f^{-1}(X)) \arrow[r, "\phi_{f^{-1}(X)}"']
& B(f^{-1}(X))
\end{tikzcd}
\]

But $f^{-1}(X) = Y$, so taking $\eta_A = \rho_{Y, Y} = id_{A(Y)}$ (likewise for $\eta_B$) implies the result, since $\phi$ is a map $A \rightarrow B$ and so must be natural wrt the restriction maps.

\newpage
4.4 (b) Show that a natural isomorphism between left exact functors \\
$\tau: T_1 \rightarrow  T_2$ induces a natural isomorphism $R^nT_1 \rightarrow R^nT_2$ for each $n$.\\

Let $T_1$ and $T_2$ be left exact functors from $\mathcal{C} \rightarrow \mathcal{D}$ and let $\tau: T_1 \rightarrow T_2$ be a natural isomorphism. Let $0 \rightarrow C \rightarrow I^0 \rightarrow I^1 \rightarrow ...$ be an injective resolution of $C$ for some $C \in \mathcal{C}$. Naturality of $\tau$ implies that the following commutes:

\[
\begin{tikzcd}
0 \arrow[r] 
& T_1(C) \arrow[r] \arrow[d, "\rotatebox{90}{\(\sim\)}"']
& T_1(I^0) \arrow[r] \arrow[d, "\rotatebox{90}{\(\sim\)}"']
& T_1(I^1) \arrow[r] \arrow[d, "\rotatebox{90}{\(\sim\)}"']
& T_1(I^2) \arrow[r] \arrow[d, "\rotatebox{90}{\(\sim\)}"']
& ...
\\
0 \arrow[r]
& T_2(C) \arrow[r]
& T_2(I^0) \arrow[r]
& T_2(I^1) \arrow[r]
& T_2(I^2) \arrow[r]
& ...
\end{tikzcd}
\]

Moreover, since $T_1$ and $T_2$ are both left exact, each row in the above diagram is exact. Then, by analogous argument to 4.3(b), $\tau$ restricts to a natural isomorphism $R^nT_1 \rightarrow R^nT_2$ for each $n$.

